\chapter{REVIEW OF RELATED LITERATURE}
{\baselineskip=2\baselineskip
This chapter highlights related projects or studies that offer valuable insights to the researchers, serving as a foundation for the development of the study.

%-----------------------------------------------------------------------------------------------------------------------
\section{ Design of a Solar-Based Portable Power Supply with Modular Battery System for the Dumagat Tribe in Norzagaray, Bulacan}

Access to reliable and sustainable energy is a critical issue in many rural and indigenous communities. One such community is the Dumagat tribe in Norzagaray, Bulacan, where there is a notable lack of access to electricity despite the usage of electronic devices like phones. Addressing this issue requires innovative solutions that not only meet their energy demands but also ensure sustainability and accessibility. 

In this context, Gozano et al. (2023) conducted a study on a solar-powered portable power supply designed to provide the Dumagat Tribe with basic energy needs. This system incorporated a solar panel, a modular battery pack, and an inverter, aimed at powering low-energy devices and addressing the tribe’s electricity needs. Using an Arduino data logger, the study measured the charging and discharging rates of the system, showing promising results while also identifying areas for improvement, such as the need for higher-capacity batteries and more efficient solar panels.

\section{Construction of a Portable Solar Power Supply for Household Appliances}

Oluwasegun et al. (2018) developed a portable solar power supply system designed to power household appliances, offering a sustainable, eco-friendly alternative to traditional non-renewable energy sources. The system consists of a solar panel, a charge controller, a 12V lead-acid battery, a pure sine wave inverter, and an Arduino phase for voltage monitoring. The system was tested with household appliances such as a 300W electric blender and a 300W electric kettle, demonstrating its ability to provide reliable power. The solar panel efficiently converts sunlight into electricity, which is stored in the battery for use when solar energy is unavailable, providing a continuous energy supply.

The portable solar power system was able to power household appliances effectively, with minimal voltage drop during use, ensuring its practicality and dependability. By using solar energy, the system reduces electricity costs and promotes environmental sustainability by lowering greenhouse gas emissions. The use of a pure sine wave inverter ensures stable and high-quality power for sensitive appliances. Oluwasegun et al. (2018) concluded that this system offers a cost-effective, clean energy solution for households, though further enhancements are needed to support higher electrical loads for more extensive applications.

\section{Emergency Solar Portable Power Supply}

The "Emergency Portable Solar Power Supply" study by Ramly et al. (2019) explored the development of a portable solar power system designed to provide electricity in areas without reliable grid access, particularly during power outages. The system utilizes solar photovoltaic (PV) technology to convert sunlight into electricity, which is stored in a battery for later use. Key components of the system include solar panels, batteries, charge controllers, inverters, and a microcontroller with Bluetooth functionality for remote monitoring. The study emphasizes the importance of sizing the solar panels and batteries properly to ensure a consistent energy supply, particularly in off-grid or emergency situations. By harnessing renewable solar energy, the system reduces reliance on fossil fuels and contributes to environmental sustainability. The solar panels used in the system have an efficiency rating of around 15-18%.

\section{Portable Solar-Station with Integrated Battery Management and Load Monitoring System}

Abdur et al. (2024) describe the development of a portable solar-powered station designed for disaster response, offering a 200W output using a 2x2 array of 50W solar panels. The system integrates a battery management system (BMS) and load monitoring mechanism (LMM), providing power for essential needs such as lighting, mobile device charging, and medical equipment in emergency situations. Replacing diesel generators reduces environmental pollution and noise, offering a sustainable, clean energy solution in areas with limited infrastructure. Its compact design allows easy transportation to disaster zones, making it an effective tool for providing reliable power where it’s needed most.

The system also features an MPPT solar charge controller, optimizing energy efficiency and extending battery life by preventing overcharging and undercharging. The design incorporates Nanton YIDA YD-W50 Monocrystalline solar panels and a 12V 80Ah lead-acid battery, ensuring reliable energy storage. Abdur et al. (2024) conclude that this portable solar station provides a cost-effective, low-maintenance, and environmentally sustainable alternative to traditional diesel generators, making it a practical solution for disaster-prone regions.

\section{Solar Powered Mobile Power Bank Systems}

Solar-powered mobile charging systems offer a sustainable solution for powering devices during power interruptions and disasters. The research by Abdur et al. (2024) focuses on a Solar-Powered Portable Power Bank designed for mobile phones, using solar energy to charge a battery, which in turn provides power through a USB port. This system is particularly useful in disaster events and remote areas with limited electricity. The system utilizes two 6V solar panels to charge a 12V battery, with a microcontroller monitoring the battery’s charge level and controlling relay circuits to ensure safe charging.  LEDs display the battery charge level, offering a user-friendly interface for monitoring.

The proposed system has multiple advantages, including reducing reliance on traditional power sources and providing a reliable charging solution in emergencies. It addresses environmental concerns by utilizing renewable energy, minimizing pollution compared to conventional power generation methods. The microcontroller ensures efficient charge flow and protects the system from damage caused by overcharging or voltage fluctuations. Abdur et al. (2024) conclude that for optimal performance, the system requires direct sunlight and proper placement of the solar panels. The design can be improved by enhancing its portability and ensuring better protection for the mobile devices and battery, making it a practical and eco-friendly alternative to traditional charging methods.

\section{Multiport Universal Solar Power Bank}

Altelmessani et al. (2024) introduce the concept of a Multiport Universal Solar Power Bank, designed to harness solar energy for a portable power supply. This device aims to address critical needs in emergency situations and recreational activities like camping, especially in remote areas where access to electricity is limited. Equipped with solar panels for efficient energy absorption, the power bank offers both DC and AC outputs, making it versatile enough to charge a wide range of electronic devices. The project emphasizes the importance of sustainability, using renewable resources to reduce reliance on non-renewable energy sources and provide a reliable, eco-friendly power source.

The system is capable of powering various appliances, including small household electronics, portable fans, lights, and mobile devices, with an AC output of up to 400W and a battery capacity of 20,000mAh. It also incorporates safety features to protect against overcharging, overheating, and other risks, ensuring both user safety and device longevity. The study highlights the project’s compact and lightweight design, making it ideal for on-the-go use, such as in emergencies or outdoor activities. By focusing on energy independence and environmental sustainability, the project contributes to reducing environmental impact while offering a reliable and versatile energy solution.

\section{Portable Power Supply Design with 100 Watt Capacity}

Zakri et al. (2021) developed a portable solar power supply design with a 100W capacity, aimed at providing sustainable energy in areas with limited electricity access. The system utilizes solar cells and a transformer to store energy in batteries with capacities of 20Ah, 60Ah, and 100Ah. The solar-powered generator can charge devices like lamps, laptops, LED televisions, and fans, supporting electrical loads under 100 watts for up to 12 hours. The system includes a Solar Charge Controller (SCC) to prevent overcharging, ensuring battery safety and longevity. The design is portable, easy to operate, and suitable for off-grid locations such as plantations or rural areas.

The tool's efficiency depends on the battery capacity and weather conditions. For example, charging the 20Ah battery takes about 5 hours under optimal sunlight, while larger batteries (60Ah and 100Ah) require more than one day to fully charge. This design is versatile, offering both AC and DC outputs, and is equipped with an LCD for voltage display, as well as safety features such as overcharge protection. Zakri et al. (2021) conclude that this portable power supply offers a practical, environmentally friendly solution for off-grid applications, with the flexibility to charge batteries via solar energy or electrical sources.

\section{Design and Development of Portable StandAlone Solar Power Generator }

Prathiba et al. (2020) developed a portable, standalone solar power generator designed to replace diesel generators with a sustainable, eco-friendly solution. The system integrates a solar panel, a battery, a bidirectional buck-boost converter, and an inverter, all supported by a Maximum Power Point Tracking (MPPT) algorithm for optimal efficiency. The generator provides a green energy source to meet load requirements and stores excess energy in a battery for use when solar energy is unavailable. The bidirectional converter enhances battery charging efficiency and ensures regulated DC voltage output, while the MPPT algorithm maximizes power extraction from the solar panel to improve overall system performance.

The portable solar generator utilizes a bi-directional converter and MPPT to achieve high efficiency, allowing it to charge and discharge a 12V lead-acid battery. The system is capable of powering both DC and AC loads, using a push-pull full-bridge inverter to drive AC devices. The system is designed for off-grid applications, including emergency situations and areas without access to electricity. The study emphasizes the system's compact design, cost-effectiveness, and potential for use in relief camps and remote locations. The project demonstrates a practical and portable renewable energy solution for sustainable power generation in diverse applications, especially in areas with limited access to the grid.

\section{A solar-powered multi-functional portable charging device (SPMFPCD) with internet-of-things (IoT)-based real-time monitoring- An innovative scheme towards energy access and management}

Rehman et al. (2024) propose a solar-powered multi-functional portable charging device (SPMFPCD) with IoT-based real-time monitoring, designed to address the growing need for reliable and versatile energy solutions across various sectors, including transportation, communication, and emergency services. The device integrates a highly efficient solar panel, a charge controller, sensors, and an IoT module for real-time monitoring of power parameters. This innovative system supports diverse applications, such as emergency medical device charging, outdoor adventures, disaster management, and public spaces. The IoT capabilities provide continuous monitoring, ensuring efficient operation and proactive maintenance, enhancing the reliability and scalability of the system.

The study emphasizes the significance of integrating advanced technologies, such as IoT-driven battery energy storage system (BESS) health monitoring, to optimize the performance and lifespan of the system. The study also conducted an economic and environmental impact assessment, showing the feasibility and sustainability of widespread SPMFPCD deployment. The proposed system demonstrated competitive cost-effectiveness, with a low cost of electricity and minimal annual operating costs. The integration of renewable energy sources like solar power and the IoT-based health monitoring system positions the SPMFPCD as a promising solution for providing accessible, environmentally friendly energy in various settings, highlighting its potential to contribute to sustainable energy management and community empowerment.

\section{Development of a Low Cost Portable Hydro and Wind Power as Emergency Power Source}

The study of low-cost portable hydro and wind power system as an emergency power source by Supardi et al. (2020) explored the development of a low-cost portable hydro and wind power system as an emergency power source. The system is designed to provide power for recharging essential equipment such as mobile phones, radios, and emergency lights, which are essential during prolonged outdoor activities or power outages. This system can be implemented in remote areas with fluctuating water flows, making it an ideal solution for communities with limited access to electricity. The study found that the portable hydropower system, when coupled with wind power, could generate reliable electricity. These findings support the integration of renewable energy sources in off-grid locations, where accessibility to traditional power infrastructure is limited, and emphasize the importance of efficient turbine and generator design for portable power generation.

\section{Development of Portable Solar Storage Device}

Roslan et al. (2019) developed a portable solar storage (PSS) device designed to address the increasing demand for sustainable and portable power sources. The PSS is tailored for outdoor activities, such as hiking, camping, and climbing, providing a portable solution for charging electronic devices. The system uses a 12V or 18V solar panel to charge a battery with 98W of energy storage. During testing, the system showed charging rates of 16.086 W/h for the 12V solar panel and 13.35 W/h for the 18V panel, demonstrating the effectiveness of solar energy harvesting. The PSS was capable of recharging devices with capacities up to 10,000mAh, offering a reliable power supply for small electronic gadgets.

The study highlights the advantages of using solar panels for off-grid applications, noting that the PSS is ideal for areas with ample sunlight, such as Malaysia. The device’s efficiency was calculated at 12.75\%, with the system capable of charging devices like phones, smartphones, and power banks. However, the study also found limitations in charging larger devices with the 12V solar panel. Roslan et al. (2019) concluded that while the PSS provides an environmentally friendly and practical solution for outdoor power needs, further improvements in solar panel capacity and battery efficiency are needed to enhance its performance for higher power demands.

\section{Renewable Energy from Solar Panels: A Study of  Photovoltaic Physics and Environmental Benefits}

Jaiswal (2023) provides an in-depth analysis of solar energy’s growing role in the global energy transition, focusing on its environmental, economic, and technological advantages. The study highlights that global solar photovoltaic (PV) capacity reached about 1,059 gigawatts by 2021, reflecting rapid adoption and its significant contribution to reducing greenhouse gas emissions. Technological advancements, such as bifacial and perovskite solar cells, have increased the efficiency and affordability of solar power, making it more accessible. The research also stresses the importance of supportive policies and regulatory frameworks in promoting solar energy deployment.

The environmental benefits of solar power are significant, with Jaiswal (2023) noting that solar energy could reduce up to 80\% of greenhouse gas emissions by 2050. Additionally, the integration of solar energy with energy storage systems is essential for improving reliability and addressing challenges related to intermittency. The study concludes that solar energy plays a crucial role in achieving sustainable development, reducing climate change impacts, and driving economic growth. With ongoing technological advancements and effective policies, solar power is set to be a key component of the future energy system.

\section{Power Consumption of Household Appliances }

Power consumption patterns in households are significantly influenced by the frequency and duration of appliance usage. Pulvera (2021) found that household appliances such as televisions, electric fans, and refrigerators are among the most frequently used devices, contributing considerably to total electricity consumption. The results revealed that the television set operates for an average of forty-four (44) hours per week, followed by the electric fan with fifty-one (51) hours, and the refrigerator with one hundred five (105) hours of usage weekly. These appliances are commonly prioritized for their essential roles in providing comfort, entertainment, and food preservation within the household.

Pulvera (2021) further emphasized that electricity usage is affected by several factors, including low voltage supply, appliance wattage, power interruptions, standby power, and user behavior. Among these, user awareness and proper energy management play a crucial role in reducing unnecessary power consumption. The continuous use of high-demand appliances such as electric fans and refrigerators, especially in tropical climates like the Philippines, highlights the dependence of households on these devices for maintaining comfort during hot weather and ensuring food preservation. Consequently, understanding the frequency of appliance use helps identify energy-saving opportunities and promote consumer awareness.

\section{Electricity Distribution and Supply Authority}

Battery selection plays a critical role in optimizing the performance and sustainability of photovoltaic (PV) systems. According to the Electricity Distribution and Supply Authority (EDSA, 2024) under the Government of Sierra Leone, three major battery options are typically considered in solar power applications—Flooded Lead Acid, Sealed Lead Acid (SLA), and Lithium Iron Phosphate (\ce{LiFePO4}). The study highlighted that \ce{LiFePO4} batteries are preferred for the Regional Emergency Solar Power Intervention (RESPITE) Project due to their superior efficiency, longevity, and environmental advantages.

\ce{LiFePO4} batteries are generally more energy-efficient, allowing for greater energy utilization and reduced losses during charging and discharging. In contrast, SLA batteries exhibit lower efficiency, which can result in higher energy wastage. Additionally, \ce{LiFePO4} batteries support a higher depth of discharge (DoD), enabling deeper energy use without significantly affecting their lifespan. This characteristic makes them more suitable for daily solar energy cycling compared to SLA batteries, which tend to degrade faster under similar conditions.

Another major advantage of \ce{LiFePO4} technology is its longer cycle life. The EDSA (2024) report notes that \ce{LiFePO4} batteries can endure more charge-discharge cycles before their performance diminishes, offering better long-term reliability. They also possess higher energy density, meaning they can store more energy in a smaller physical footprint—an important factor when space constraints exist in solar installations.

In terms of maintenance, \ce{LiFePO4} batteries are maintenance-free, unlike SLA batteries which often require periodic electrolyte checks and refilling. While \ce{LiFePO4} batteries have a higher upfront cost, they compensate with lower operational and maintenance expenses throughout their lifespan. From an environmental perspective, \ce{LiFePO4} batteries are also less harmful and easier to dispose of or recycle compared to lead-acid types, which generate toxic emissions during recycling.

\section{Assessing the Impact of Power Outages on Appliances of Farmers and Fisherfolks in Selected Barangays of Cawayan, Masbate, Philippines: Basis for a Proposed Extension Program}

Frequent power interruptions have long been a challenge in rural communities, particularly among farmers and fisherfolk who depend on electricity for household and livelihood activities. The study aimed to determine the impact of frequent power interruptions on the appliances and economic well-being of residents in selected barangays of Cawayan, Masbate. Employing a descriptive research design, the researchers gathered data from 266 respondents through paper surveys and face-to-face interviews to assess the frequency, duration, and effects of power interruptions. Findings revealed that almost all respondents experienced power interruptions lasting three to four hours, leading to increased electricity consumption and higher bills. Refrigerators and televisions were the most power-consuming appliances, and there was significant damage to appliances, especially bulbs, as well as disruptions to income-generating activities. The study results showed that many respondents had an annual income of less than \textpeso18,200, which was considered low and may have resulted in difficulty in paying high bills brought by power outages. All respondents relied on the power grid as their source of electricity, and power interruptions were a common occurrence. The data revealed that 97.7\% of respondents experienced power interruptions, with 51.1\% experiencing 3-4 hours of interruption. Almost all respondents claimed that power interruption increased their electric consumption and bill, and 56\% were not satisfied with their electric bill when there was a power interruption. It was concluded that unreliable electricity supply and lack of maintenance significantly affect low-income households that rely solely on the power grid as their source of electricity. The study recommended strengthening local power infrastructure and promoting renewable energy education through a proposed extension project that trains communities in the use of solar energy as a sustainable and reliable alternative to the current power grid system. This helps reduce air pollution and climate change while building the capacity of farmers and fishers to adapt to these changes.

\section{Solar-Powered Coin-Operated Mobile Charging Station for Sustainable Energy Access and Resilience}

The increasing demand for sustainable and accessible energy has driven innovations that utilize renewable sources to meet electricity needs in both urban and rural areas. The study by Catalan, Occeña, Gabion, Occeña, and Ejar (2023) aimed to develop a solar-powered, coin-operated mobile charging station designed to provide continuous off-grid power using photovoltaic (PV) technology. Employing a developmental research approach, the researchers designed, fabricated, and tested a prototype equipped with solar panels and an integrated storage battery system capable of charging multiple mobile devices for both commercial and emergency purposes. The results demonstrated that the charging station effectively powered various mobile gadget models with no compatibility issues, maintained stable operation even under limited sunlight, and offered cost efficiency through its low maintenance and sustainable design. Furthermore, the system contributed to reducing carbon emissions and supported the green technology initiatives of Guimaras State University. It was concluded that the solar-powered charging station is a practical and eco-friendly innovation that promotes energy resilience in communities affected by power outages. The study recommended installing such systems in strategic locations for communal purposes and remote areas, integrating security features to prevent misuse, and conducting further research to enhance its technological design and promote the widespread adoption of renewable energy solutions.

\section{Synthesis}

The reviewed literature collectively highlighted the potential of solar-powered systems such as several power sources, mostly portable power stations as sustainable energy solutions, especially in off-grid and rural areas where power interruptions are common. Studies by Gozano et al. (2023), Oluwasegun et al. (2018), and Ramly et al. (2019) developed solar-based portable power supplies that integrated photovoltaic panels, charge controllers, batteries, and inverters. While Rehman et al. (2024) integrated IoT for real-time monitoring and efficiency optimization, later research by Abdur et al. (2024) and Altelmessani et al. (2024) enhanced system functioning through battery management and load monitoring systems. The suggested system’s use of an ESP32 microcontroller, GPS and GSM modules, and LCD-based monitoring for energy tracking and operational management directly demonstrates these works’ focus on  the importance of effective energy harvesting, battery management, inverter stability, and smart control for system reliability.

The importance of choosing appropriate energy storage methods is also emphasized by a number of studies. Lithium Iron Phosphate (LiFePO4) batteries are preferred to conventional lead-acid batteries because of their longer cycle life, higher efficiency, and maintenance-free operation, which makes them suitable for solar-powered systems with frequent charge-discharge cycles in accordance to EDSA (2024), and Zakri et. al (2021). Moreover, the Masbate (2024) study and Pulvera (2021) illustrate how heavily Filipino households depend on devices like fans, televisions, and refrigerators which are often affected by power interruptions. 


This insight emphasizes the necessity of accessible and reasonably priced renewable energy sources that can sustain necessary home loads. Moreover, Catalan et al. (2023) demonstrated that coin-operated solar-powered charging stations effectively promote energy access and sustainability, supporting the integration of a coin-slot mechanism and solenoid lock for secure, pay-per-use operations in community-based systems.

A significant study gap still exists in the integration of several key elements into a single deployable system, even if these studies collectively validate the effectiveness and feasibility of portable solar systems.	An IoT-enabled monitoring and tracking system, a rental-based model with detachable power sources, and secure access mechanisms intended for community users are not offered in the literature currently in publication. A 2x2 solar panel array, an MPPT charge controller, a LiFePO4 battery for effective energy storage, and an inverter for AC conversion are all integrated into the proposed Solar-Powered Rental Station with Detachable Power Sources to address the problem at hand. 

The system integrates ESP32-based monitoring, GSM and GPS connectivity, and a coin-slot mechanism for secure and traceable rentals, with a mobile application for real-time battery monitoring. By combining sustainable energy harvesting and smart control, it offers a practical and eco-friendly solution to power interruptions in communities. In summary, the synthesis of current literature highlights the importance of integrating efficient solar energy systems, smart monitoring, reliable storage, and accessible user feature concepts that form the foundation of the proposed Solar-Powered Rental Station with Detachable Power Sources.



}