\chapter{INTRODUCTION}

{\baselineskip=2\baselineskip


\section{Background of the Study}
Electricity is essential to everyday living, powering household activities from lights and appliances to connectivity. It is the most used energy source and a notable expense among Filipino households, with a 94.8\% usage rate (Philippine Statistics Authority, 2023). Frequent power interruptions disrupt comfort, work, and livelihoods, forcing households to rely on diesel generators as temporary backup. However, these generators are costly and contribute to noise and air pollution, highlighting the need for alternative, sustainable power sources.

Research has proven that solar-powered systems offer a reliable and eco-friendly power source. Catalan et al. (2023) developed a solar-powered coin-operated mobile charging station designed that provides off-grid charging for mobile devices through photovoltaic panels and integrated battery storage. Their study emphasized the system’s effectiveness in extending energy access to remote and island communities, particularly in locations where electricity is limited. Moreover, Catalan et al. (2023) also emmphasized the importance of incorporating additional security features to ensure the consistent and reliable operation of such systems over time. Similarly, photovoltaic (PV) solar energy is becoming a key player in the global energy transition, providing renewable and scalable energy solutions (Sampaio \& González, 2017; Ramly, Jamal, Abd Ghafar, \& Babu, 2019).

Technology also plays a vital role in improving such systems. According to Villamil et al. (2020), the Internet of Things (IoT) is an emerging technology present in many devices and processes, improving quality of life. Integrating IoT with solar energy systems enables smart monitoring, management, and accessibility. Rental-based systems can further improve affordability and accessibility. Thakur (2021) illustrated that a car rental system efficiently manages transactions, user data, and resources through database-driven modules and web-based interfaces. Applying similar principles to energy rentals allows automation of verification, monitoring, and payment processes through an IoT-based application. In support of this, Gavera et al. (2024) highlighted the relevance of the ESP32 as a powerful microcontroller for developing efficient and stable sensor networks, enhancing sustainability and resource efficiency. Combining these innovations can maximize utilization, enhance system functionality, and promote sustainable energy access for communities affected by frequent power interruptions.

In the Philippines, the average household experiences 28 electricity supply interruptions per year due to outages caused by plant breakdowns (Albay, 2025). In August 2025, the National Grid Corporation of the Philippines (NGCP) reported an unscheduled power interruption affecting Camiguin and Misamis Oriental, which together house over one million households (Philippine Statistics Authority, 2020). During such outages, reliance on diesel generators and other temporary energy sources underscores the inadequacy of existing solutions and the urgent need for reliable, affordable, and sustainable alternatives.

Recent efforts in solar-powered portable power supply systems show promise but still face limitations. Innovations have been made, such as Gozano et al. (2023) designing a solar-based portable power supply capable of charging small devices but with limited capacity and short battery life. Bhatti et al. (2024) developed a portable solar station with a 200 W output and load monitoring, while Ramly et al. (2019) created a 100 W portable solar power supply for emergencies that could last two days. Despite these advancements, existing systems still encounter issues related to energy capacity, battery lifespan, and limited support for higher power loads.

To address these gaps, this study proposes a solar-powered rental station with detachable power sources, powered by a solar panel for energy harvesting, a LiFePO4 battery for reliable storage, and an inverter for AC power conversion. The system integrates an ESP32 microcontroller with IoT mechanisms to manage operations, GPS for location tracking, and GSM for communication. A coin-slot mechanism with a solenoid lock ensures secure access, while the mobile application allows users to view available rental slots, check pricing based on the detachable power source’s charge percentage, and manage user verification through SMS-based phone number confirmation. The application also stores the user’s account information and available credits for future transactions. The compact and user-friendly design offers a sustainable solution for communities affected by frequent power interruptions, effectively bridging the gap in affordable and reliable energy access.

\section{Statement of the Problem}

Communities in various parts of the Philippines continue to face challenges in accessing stable and reliable electricity. In provinces such as Camiguin (served by CAMELCO) and parts of Misamis Oriental (served by MORESCO II), residents experience frequent and unscheduled power interruptions multiple times a week. These power outages cause inconvenience to households, affect business operations, and disrupt the productivity of local communities. Consistent and stable access to electricity remains a persistent challenge, especially in rural and island areas.

Although alternatives such as off-grid systems, portable power banks, and small gasoline generators are available, they often prove costly, limited in capacity, or unsustainable in the long term. The use of fossil-fueled generators also raises safety and environmental concerns, which contradict current efforts toward cleaner and more sustainable energy sources.

Solar-powered systems have been introduced as alternatives, yet most existing systems can only supply limited power to small devices such as mobile phones, lamps, or AM/FM radios. Many of these systems are installed within college campuses and are not designed for community accessibility. In addition, the absence of security and monitoring mechanisms makes these systems prone to tampering, theft, and misuse.

To address these challenges, this study aims to answer the following research questions:

\begin{enumerate}
	\item  How can a solar-powered rental station with detachable power sources be designed to provide a sustainable and user-friendly energy solution for communities experiencing frequent power interruptions?
	\item What system architecture can effectively integrate photovoltaic panels, energy storage, and power inverters to ensure sufficient capacity for small to medium-scale appliances?
	\item How can a secure coin-operated or mobile application-based mechanism be developed to regulate and monitor the rental and return of detachable power units?
	\item How can GPS and GSM technologies be integrated to enhance real-time tracking, monitoring, and communication of rented power sources?
	\item To what extent can the developed prototype address issues of affordability, accessibility, and sustainability compared to existing backup power solutions?
	\item To what extent can the proposed system overcome existing limitations in solar-powered portable power solutions, specifically lower energy load, limited battery capacity, restricted support for higher loads, 
\end{enumerate}

\section{Objectives of the Study}

In view of the above stated problem, the following objectives are:

\subsection{General Objectives}
\begin{itemize}
	\item To design and develop a solar-powered portable power station with detachable power sources that are rented to serve as a sustainable and user-friendly solution for supplying electricity, specifically intended for communities experiencing frequent and unscheduled power interruptions. 
\end{itemize}
\subsection{Specific Objectives}
\begin{itemize}
	\item To design the system architecture for the portable power station that integrates photovoltaic panels, energy storage, and a power inverter, ensuring sufficient capacity for powering small to medium-scale appliances.
	\item To develop a functional prototype of the portable power station with a detachable power source, with a secure coin-operated or app-based access mechanism for rental use.
	\item To integrate GPS and GSM modules to support real-time location tracking, communication, and monitoring of rented units.
\end{itemize}

\section{Significance of the Study}

This project benefits a diverse range of stakeholders including: 

\vspace{0.6em} 
\addtocounter{subsection}{1}
\noindent\textbf{\thesubsection\ Community Residents:} 
Community residents will benefit from continuous access to electricity through the solar-powered portable power source. This system ensures the community residents can use essential household appliances and stay connected during power interruptions, improving their daily lives and overall comfort.

\vspace{0.6em} 
\addtocounter{subsection}{1}
\noindent\textbf{\thesubsection\ Environment:} 
This study contributes to environmental sustainability by promoting the use of solar energy, a renewable resource, instead of relying on fossil-fueled generators. The system reduces carbon emissions and pollution, aligning with global sustainability goals.

\vspace{0.6em} 
\addtocounter{subsection}{1}
\noindent\textbf{\thesubsection\ Future Researchers:} 
Future researchers will find this study helpful because it gives important information on how to create solar-powered portable charging systems with features like GPS tracking, coin/app payment, and real-time monitoring. 

\vspace{0.6em} 
\addtocounter{subsection}{1}
\noindent\textbf{\thesubsection\ SDG 7: Affordable and Clean Energy:} 
This system provides support in giving people access to energy that is affordable, reliable, and environmentally friendly by offering a solar-powered rental station that provides communities, especially those often experiencing blackouts, with a clean and low-cost source of electricity.

\vspace{0.6em} 
\addtocounter{subsection}{1}
\noindent\textbf{\thesubsection\ Industry, Innovation, and Infrastructure:} 
 This helps achieve the goal of improving systems, creating new technologies, and building stronger infrastructure by using modern technologies, as well as solar energy, which helps in creating innovative solutions that give communities a reliable energy option during power interruptions.


\section{Scope and Limitations}

\subsection{Scope}
	\begin{itemize}
		\item Design a solar-powered portable power station with detachable power sources for communities in Camiguin and selected areas in Misamis Oriental, which face frequent power interruptions.
		\item Integrate photovoltaic panels, energy storage, and a power inverter to run small- to medium-scale appliances.
		\item Implement coin-operated or app-based access for renting the power sources.
		\item Incorporate GPS and GSM modules for real-time location tracking and communication, ensuring efficient monitoring of rented units.
\end{itemize}

\subsection{Limitations}
	\begin{itemize}
		\item  Geographic coverage limited to Camiguin and selected areas in Misamis Oriental.
		\item  Not capable of operating high-demand power appliances due to limited power capacity.
		\item Cannot charge electric vehicles (e.g., Tesla) because required input is far higher than the system can supply.
		\item Deployment limited to suitable sites: Only for selected areas that both (a) frequently experience power interruptions and (b) have adequate sunlight; performance is poor where solar access is obstructed (c) inoperable in areas without Internet connectivity
		\item Coins only: The coin-operated mechanism does not accept paper bills.
		\item No physical change: The system does not dispense change; any excess payment/remaining balance is credited to the user’s mobile application account.
		\item Weather-dependent operations: The system cannot operate during extreme weather, such as typhoons to protect users and equipment from potential damage
	\end{itemize}

\subsection{Definition of Terms}


\begin{itemize}
	\item Internet of Things (IoT) - A network of physical devices, such as sensors, appliances, and power sources, connected to the internet, enabling them to collect, send, and receive data for remote monitoring, control, and management.
	
	\item Solar Energy - Energy that is harnessed from sunlight using technologies such as photovoltaic (PV) panels, which convert sunlight into electricity, which is used to power the portable power sources.
	
	\item Fossil Fuel - Natural energy sources such as coal, oil, and natural gas, derived from the remains of ancient plants and animals, that are burned to produce energy but contribute to environmental pollution and climate change due to the emission of greenhouse gases.
	
	\item Photovoltaic (PV) Panels - Solar panels that convert sunlight into electricity, serving as the primary source of power generation for solar-powered systems.
	
	\item Portable Power Station - Compact, mobile units that provide electrical power for charging devices or operating appliances, often powered by renewable sources like solar energy and designed to be easily transported or moved.
	
	\item Renewable Energy - Energy derived from natural resources that are replenished on a human timescale, such as sunlight, wind, and geothermal heat, which are harnessed to produce electricity in an environmentally sustainable manner.
	
	\item Off-grid power system - A power system that operates independently from the main electricity grid, using renewable energy sources like solar or wind to provide electricity in areas without access to centralized power.
	
	\item Solar harvesting - The process of capturing sunlight using solar panels or other solar technologies and converting it into usable electrical energy, typically for storage in batteries or direct use.
	
	\item Power Interruption - A temporary loss or disruption of electrical power, often due to faults, maintenance, or other technical issues in the power grid, affecting the availability of electricity to households or businesses.

\end{itemize}

}
