\chapter{INTRODUCTION}

{\baselineskip=2\baselineskip


\section{Background of the Study}

Electricity is essential to everyday living, powering different activities and circumstances, especially in households, from lights and appliances to connectivity.  Electricity powers nearly every aspect of home living, being the most used energy source and notable expense among Filipino households, with a 94.8\% usage rate based on the Philippines Statistics Authority (2023). During frequent power interruptions, Filipino households face severe impacts from frequent power interruptions including comfort issues and disruption of work and livelihoods.  Many households rely on diesel generators as a temporary measure, but these are costly to run and contribute to noise and air pollution, indicating the need for other alternative measures for power sources.

According to Villamil et. al (2020), “The internet of things is an emerging technology that is currently present in most processes and devices, allowing to improve the quality of life of people.” As IoT continues to revolutionize daily living, solar energy, powered by innovations such as photovoltaic (PV) panels, plays a crucial role in shaping the future of global energy. Photovoltaic solar energy is rapidly becoming a key player in the global energy transition. This surge in demand and innovation paves the way for a more sustainable future, making solar energy a central focus in the race for renewable solutions. As highlighted by Sampaio and González (2017), "The capture of solar energy through photovoltaic (PV) panels to generate electricity has emerged as one of the most promising markets in renewable energy.”  In response to this, many systems utilize off-grid renewable energy; systems such as solar power systems are designed to provide electricity. This aligns with the study of Ramly, Jamal, Abd Ghafar, and Babu (2019), who highlighted that the “Emergency Portable Solar Power Supply is a product that uses a renewable energy source (sunlight) as the main source of electricity.”

Today's solar power systems offer many benefits. The technologies mentioned above can contribute to addressing key issues in the Philippines. They aim to reduce the impact of frequent power interruptions and emergencies, while promoting scalable renewable energy innovations for affected communities. Additionally, it contributes to sustainable energy systems by addressing the unique reliability challenges of off-grid and disaster-prone areas in the region.

 According to an article by Albay (2025), In the Philippines, the average Filipino household experiences 28 electricity supply interruptions in a year, due to frequent outages caused by plant breakdowns based on a report. The sudden or frequent power interruptions force households and businesses alike to rely on costly diesel generators as a backup power source. In August  2025, the National Grid Corporation of the Philippines (NGCP) reported an unscheduled power interruption affecting Camiguin and Misamis Oriental, which are home to 92,808 and 956,900 households based on the Philippine Statistics Authority (2020). The outage left thousands of households and businesses without power, mostly relying on diesel generators and various power sources. Addressing this gap in reliable energy access is crucial to providing communities with an affordable, sustainable power alternative and reducing the impacts of sudden interruptions. 

 Recent efforts to adapt solar-powered power supply systems show potential but face limitations. For instance, Gozano et al. (2023) designed a solar-based portable power supply with a modular battery pack. However, the portable power supply was shown capable of only supplying a lower energy load, such as charging mobile phones, mobile lighting and small auxiliary loads such as a small AM/FM radio. There were also shortcomings, such as the short lifespan of AC power whenever the battery pack reaches a certain voltage drop. 
 
 Innovations in solar-powered power supply systems offer a promising solution to these challenges. Bhatti et al. (2024) developed a portable solar station with integrated battery management and load monitoring system, having a power output of 200 W, and a 2x2 array of 50 W solar panels to provide electricity for basic necessities, in times of power interruptions due to natural disasters. Similarly,  the study by Ramly et al (2019) designed a portable solar power supply for emergency situations, supplying electricity up to 100W at one time, and utilizing Arduino Uno with Bluetooth module and voltage sensor to get the voltage readings integrated into a mobile application.  The said system lasted for two days without the need for charging. Despite these innovations, current solar-powered power supply systems still face limitations, such as low energy capacity, short battery lifespan, and restricted support for higher power loads.
 
 To bridge these gaps, this study proposes a solar-powered rental station for detachable power sources with a 2x2 solar panel array for energy harvesting, a LiFePO4 battery for reliable storage, and an inverter for AC power conversion. The system integrates an ESP32 microcontroller, integrating IoT mechanisms to manage operations, with GPS for location tracking and GSM for communication. A coin-slot mechanism with a solenoid lock ensures secure access, while a mobile app provides real-time monitoring of battery charge levels and system status. The compact and user-friendly design offers a sustainable solution for communities affected by frequent power interruptions.


\section{Statement of the Problem}

Air pollution is among the Philippines' most critical environmental issues, primarily caused by the use of fossil fuels. The Philippines' reliance on fossil fuels for electricity generation, particularly coal, contributes to elevated levels of air pollution. In 2025, coal accounted for approximately 52.7\% of the country's electricity generation, while natural gas contributed 20.1\%. This dependence on fossil fuels leads to the emission of pollutants that adversely affect air quality and public health. (LowCarbonPower, 2025). 

Electricity is a necessity for modern life. In Camiguin (served by CAMELCO) and certain parts of Misamis Oriental (served by MORESCO 2), the local community frequently experiences unscheduled power interruptions, occurring multiple times a week. This issue is consistent with research indicating that power interruptions are common in developing countries (Ibañez, 2024; Taniguchi, 2019), such as the Philippines, particularly in rural areas (Ibañez, 2024; Ali, 2016). These interruptions not only hinder productivity but also affect daily comfort in many households, making them very inconvenient and a source of frustration.

Despite the availability of off-grid setups, power banks, and small gasoline generators, these options are often limited by cost, power capacity, or environmental sustainability.  Furthermore, the reliance on fossil-fueled alternatives poses safety risks and contributes to pollution. In response, the government encourages a shift away from fossil fuel energy sources (Koons, 2024) and promotes the use of renewables, as outlined in the policy framework provided for in Republic Act (RA) No. 9513 or the “Renewable Energy Act of 2008”. Hence, there is a growing need for technological systems, such as solar-powered systems, to align with sustainability mandates. Studies show that the transition to greater use of renewables has wide-ranging implications (Villanueva, 2021), underscoring the importance of integrating supportive technologies into local environmental strategies.

Solar-powered systems have been introduced either as portable power supplies or as charging stations, yet existing systems face multiple limitations. Many of these are only capable of supplying low-energy loads, such as charging mobile phones, providing mobile lighting, or powering small auxiliary devices like AM/FM radios (Gozana et al., 2023). Additionally, the average cost of an off-grid setup is high (Boodoo, 2024), making it unaffordable for many households. Research also shows that solar-powered systems are commonly installed within the premises of the college campus (Catalan et al., 2019), but have not been widely initiated for communal purposes (Catalan et al., 2023). Furthermore, a notable gap exists in security, as many systems lack sufficient measures to prevent delinquent users from tampering with or vandalizing the system. These gaps, along with the promotion of solar-powered coin-operated charging stations in remote and island barangays, are highlighted by Catalan et al. (2023), as these areas often face significant challenges in power accessibility.

Given these limitations, there is a clear need for an innovative solar-powered rental system that is both affordable and scalable, while providing reliable energy access for households and public areas affected by frequent and unscheduled power interruptions. This study aims to design and develop a solar-powered rental station with detachable power sources, offering a sustainable and user-friendly alternative to conventional backup solutions, while also addressing the need for improved security features.

\section{Objectives of the Study}

In view of the above stated problem, the following objectives are:
\subsection{General Objectives}
\begin{itemize}
	\item To design and develop a solar-powered portable power station with detachable power sources that are rented to serve as a sustainable and user-friendly solution for supplying electricity, specifically intended for communities experiencing frequent power interruptions. 
\end{itemize}
\subsection{Specific Objectives}
\begin{itemize}
	\item To design the system architecture for the portable power station that integrates photovoltaic panels, energy storage, and a power inverter, ensuring sufficient capacity for powering small to medium-scale appliances.
	\item To develop a functional prototype of the portable power station with a detachable power source, with a secure coin-operated or app-based access mechanism for rental use.
	\item To integrate GPS and GSM modules to support real-time location tracking, communication, and monitoring of rented units.
\end{itemize}

\section{Significance of the Study}

This project benefits a diverse range of stakeholders including: 

\vspace{0.6em} 
\addtocounter{subsection}{1}
\noindent\textbf{\thesubsection\ Community Residents:} 
Community residents will benefit from continuous access to electricity through the solar-powered portable power source. This system ensures the community residents can use essential household appliances and stay connected during power interruptions, improving their daily lives and overall comfort.

\vspace{0.6em} 
\addtocounter{subsection}{1}
\noindent\textbf{\thesubsection\ Environment:} 
This study contributes to environmental sustainability by promoting the use of solar energy, a renewable resource, instead of relying on fossil-fueled generators. The system reduces carbon emissions and pollution, aligning with global sustainability goals.

\vspace{0.6em} 
\addtocounter{subsection}{1}
\noindent\textbf{\thesubsection\ Future Researchers:} 
Future researchers will find this study helpful because it gives important information on how to create solar-powered portable charging systems with features like GPS tracking, coin/app payment, and real-time monitoring. 

\vspace{0.6em} 
\addtocounter{subsection}{1}
\noindent\textbf{\thesubsection\ SDG 7: Affordable and Clean Energy:} 
This system provides support in giving people access to energy that is affordable, reliable, and environmentally friendly by offering a solar-powered rental station that provides communities, especially those often experiencing blackouts, with a clean and low-cost source of electricity.

\vspace{0.6em} 
\addtocounter{subsection}{1}
\noindent\textbf{\thesubsection\ Industry, Innovation, and Infrastructure:} 
 This helps achieve the goal of improving systems, creating new technologies, and building stronger infrastructure by using modern technologies, as well as solar energy, which helps in creating innovative solutions that give communities a reliable energy option during power interruptions.


\section{Scope, Limitations and Definition of Terms}

\subsection{Scope}
	\begin{itemize}
		\item Design a solar-powered rental station with detachable power sources for communities in Camiguin and selected areas in Misamis Oriental, which face frequent power interruptions.
		\item Integrate photovoltaic panels, energy storage, and a power inverter to run small- to medium-scale appliances.
		\item Implement coin-operated or app-based access for renting the power sources.
		\item Incorporate GPS and GSM modules for real-time location tracking and communication, ensuring efficient monitoring of rented units.
\end{itemize}

\subsection{Limitations}
	\begin{itemize}
		\item  Geographic coverage limited to Camiguin and selected areas in Misamis Oriental.
		\item  Not capable of operating high-demand power appliances due to limited power capacity.
		\item Cannot charge electric vehicles (e.g., Tesla) because required input is far higher than the system can supply.
		\item Deployment limited to suitable sites: Only for selected areas that both (a) frequently experience power interruptions and (b) have adequate sunlight; performance is poor where solar access is obstructed.
		\item Coins only: The coin-operated mechanism does not accept paper bills.
		\item No physical change: The system does not dispense change; any excess payment/remaining balance is credited to the user’s mobile application account.
		\item Weather-dependent operations: The system cannot operate during extreme weather, such as typhoons to protect users and equipment from potential damage
	\end{itemize}

\subsection{Definition of Terms}


\begin{itemize}
	\item Internet of Things (IoT) - A network of physical devices, such as sensors, appliances, and power sources, connected to the internet, enabling them to collect, send, and receive data for remote monitoring, control, and management.
	
	\item Solar Energy - Energy that is harnessed from sunlight using technologies such as photovoltaic (PV) panels, which convert sunlight into electricity, which is used to power the portable power sources.
	
	\item Fossil Fuel - Natural energy sources such as coal, oil, and natural gas, derived from the remains of ancient plants and animals, that are burned to produce energy but contribute to environmental pollution and climate change due to the emission of greenhouse gases.
	
	\item Photovoltaic (PV) Panels - Solar panels that convert sunlight into electricity, serving as the primary source of power generation for solar-powered systems.
	
	\item Portable Power Station - Compact, mobile units that provide electrical power for charging devices or operating appliances, often powered by renewable sources like solar energy and designed to be easily transported or moved.
	
	\item Renewable Energy - Energy derived from natural resources that are replenished on a human timescale, such as sunlight, wind, and geothermal heat, which are harnessed to produce electricity in an environmentally sustainable manner.
	
	\item Off-grid power system - A power system that operates independently from the main electricity grid, using renewable energy sources like solar or wind to provide electricity in areas without access to centralized power.
	
	\item Solar harvesting - The process of capturing sunlight using solar panels or other solar technologies and converting it into usable electrical energy, typically for storage in batteries or direct use.
	
	\item Power Interruption - A temporary loss or disruption of electrical power, often due to faults, maintenance, or other technical issues in the power grid, affecting the availability of electricity to households or businesses.

\end{itemize}

}
